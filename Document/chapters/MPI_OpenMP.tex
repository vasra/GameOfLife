\section{Επεξήγηση κώδικα}
\begin{multicols}{2}
Ο κώδικας παραμένει, στο μεγαλύτερο μέρος του, ίδιος με αυτόν που περιγράφηκε στο προηγούμενο κεφάλαιο. Έχουν γίνει μόνο κάποιες προσθήκες ώστε να χρησιμοποιούνται OpenMP threads στους υπολογισμούς. Συγκεκριμένα, στις συναρτήσεις \mintinline{C}{Initial_state} και \mintinline{C}{Next_generation_inner} έχει προστεθεί η εξής εντολή πριν το διπλό \mintinline{C}{for}: \\
\end{multicols}

\begin{tcolorbox}
\mintinline{C}{#pragma}\mintinline{C}{ omp parallel for collapse(2) private(i,j) schedule(static) reduction(+:lsum)}
\end{tcolorbox}
\begin{multicols}{2}
Η εντολή αυτή κάνει collapse  το διπλό \mintinline{C}{for} ώστε να κατανεμηθούν οι υπολογισμοί σε πολλά νήματα, κι επίσης κάνει reduce το τοπικό άθροισμα στη μεταβλητή \mintinline{C}{lsum}, η τιμή της οποίας στο τέλος περνά στην \mintinline{C}{local_sum}. \par

Στην συνάρτηση \mintinline{C}{Next_generation_outer}, έχουν προστεθεί τέσσερις νέες εντολές, μία πριν από κάθε \mintinline{C}{for} που υπολογίζει άλω σειρά ή στήλη. \\
\end{multicols}

\begin{tcolorbox}
\mintinline{C}{#pragma}\mintinline{C}{ omp parallel for schedule(static) reduction(+:lsum)}
\end{tcolorbox}
Μαζί με την παραλληλοποίηση σε νήματα, το κάθε \mintinline{C}{for} συνεισφέρει στο άθροισμα \mintinline{C}{lsum}, το οποίο στο τέλος προστίθεται στο \mintinline{C}{local_sum}.

\section{Εύρεση βέλτιστου συνδυασμού διεργασιών -- νημάτων σε έναν κόμβο}

Ακολουθεί πίνακας με τα αποτελέσματα των διάφορων συνδυασμών διεργασιών -- νημάτων. Οι δοκιμές έγιναν για μέγεθος πίνακα $840$ και $2000$ επαναλήψεις.

\begin{table}[h]
\centering
\begin{tabular}{| c | c | }
\hline
Διεργασίες--νήματα & Χρόνος \\
\hline
\end{tabular}
\caption{Συνδυασμοί διεργασιών -- νημάτων σε έναν κόμβο, και οι αντίστοιχοι χρόνοι εκτέλεσης}
\label{tab:OpenMPOneNode}
\end{table}
\section{Μετρήσεις στην Αργώ}