\section{Επεξήγηση κώδικα}
\begin{multicols}{2}
Ο κώδικας παραμένει, στο μεγαλύτερο μέρος του, ίδιος με αυτόν που περιγράφηκε στο προηγούμενο κεφάλαιο. Έχουν γίνει μόνο κάποιες προσθήκες ώστε να χρησιμοποιούνται OpenMP threads στους υπολογισμούς. Συγκεκριμένα, στις συναρτήσεις \mintinline{C}{Initial_state} και \mintinline{C}{Next_generation_inner} έχει προστεθεί η εξής εντολή πριν το διπλό \mintinline{C}{for}: \\
\end{multicols}

\begin{tcolorbox}
\mintinline{C}{#pragma}\mintinline{C}{ omp parallel for collapse(2) private(i,j) schedule(static) reduction(+:lsum)}
\end{tcolorbox}
\begin{multicols}{2}
Η εντολή αυτή κάνει collapse  το διπλό \mintinline{C}{for} ώστε να κατανεμηθούν οι υπολογισμοί σε πολλά νήματα, κι επίσης κάνει reduce το τοπικό άθροισμα στη μεταβλητή \mintinline{C}{lsum}, η τιμή της οποίας στο τέλος περνά στην \mintinline{C}{local_sum}. \par

Στην συνάρτηση \mintinline{C}{Next_generation_outer}, έχουν προστεθεί τέσσερις νέες εντολές, μία πριν από κάθε \mintinline{C}{for} που υπολογίζει άλω σειρά ή στήλη. \\
\end{multicols}

\begin{tcolorbox}
\mintinline{C}{#pragma}\mintinline{C}{ omp parallel for schedule(static) reduction(+:lsum)}
\end{tcolorbox}
Μαζί με την παραλληλοποίηση σε νήματα, το κάθε \mintinline{C}{for} συνεισφέρει στο άθροισμα \mintinline{C}{lsum}, το οποίο στο τέλος προστίθεται στο \mintinline{C}{local_sum}.

\section{Εύρεση βέλτιστου συνδυασμού διεργασιών -- νημάτων σε έναν κόμβο}

Ακολουθεί πίνακας με τα αποτελέσματα των διάφορων συνδυασμών διεργασιών -- νημάτων. Οι δοκιμές έγιναν για μέγεθος πίνακα $3360$ και $2000$ επαναλήψεις, με timeout ενός λεπτού.

\begin{table}[H]
\centering
\begin{tabular}{| c | c | }
\hline
Διεργασίες--νήματα & Χρόνος \\
\hline
1--1 & Timeout \\
1--2 & Timeout \\
1--4 & Timeout \\
1--8 & Timeout \\
1--16 & Timeout \\
2--4 & Timeout \\
2--8 & Timeout \\
2--16 & Timeout \\
\textbf{4--2} & \textbf{23.89} \\
4--4 & 52.89 \\
4--8 & 33.31 \\
4--16 & 34.79 \\
8--2 & 49.03 \\
8--4 & Timeout \\
8--8 & 25.61 \\
8--16 & 28.48 \\
\hline
\end{tabular}
\caption{Συνδυασμοί διεργασιών -- νημάτων σε έναν κόμβο, και οι αντίστοιχοι χρόνοι εκτέλεσης. Ο συνδυασμός 4--2 προσφέρει το καλύτερο αποτέλεσμα.}
\label{tab:OpenMPOneNode}
\end{table}

Οι καλύτερες επιδόσεις επετεύχθησαν με 4 διεργασίες και 2 νήματα ανά διεργασία. Επομένως, οι μετρήσεις στην Αργώ θα γίνουν με αυτές τις τιμές.

\section{Μετρήσεις στην Αργώ}
\begin{multicols}{2}
Για τις μετρήσεις στην Αργώ, επιλέγουμε συνδυασμούς διεργασιών--νημάτων από τους οποίους προκύπτουν τέλεια τετράγωνα. Επειδή βρήκαμε ότι οι αριθμοί διεργασιών--νημάτων που προσφέρουν τις καλύτερες μετρήσεις είναι και οι δύο άρτιοι (4 και 2 αντίστοιχα), δε γίνεται να κάνουμε δοκιμές για τα περιττά τετράγωνα 9, 25, 49. Επίσης, δεν θα πάρουμε ξανά μετρήσεις για 1 διεργασία με 1 νήμα, αφού το αποτέλεσμα είναι το ίδιο με αυτό στα καθαρά MPI. Για την περίπτωση αυτή, θα χρησιμοποιήσουμε ξανά τους χρόνους που δίνονται στον πίνακα \ref{tab:timesMPIAllreduce}. Επίσης, αυτή την φορά δεν θα επιτρέψουμε να τρέχει η κάθε μέτρηση παραπάνω από ένα λεπτό, προκειμένου να αποφευχθεί ο συνωστισμός στην Αργώ. Εάν ξεπεραστεί το όριο, στον πίνακα θα υπάρχει απλώς η ένδειξη ``Timeout``. Φυσικά, όπου παρατηρηθεί επιβράδυνση, σταματούν οι μετρήσεις και τα υπόλοιπα κελιά της ίδιας σειράς παίρνουν την τιμή «Ε».
\end{multicols}

\begin{table}[H]
\centering
\small
\begin{tabular}{| l | c | c | c | c | c | c | c | c | c |}
\hline
\diagbox{Μέγεθος}{Διεργασίες * Νήματα} & 4 & 16 & 36 & 64 & 80\\
\hline
840 & 5.96 & 0.82 & 0.63 & 0.56 & 0.57 \\
\hline
1680 & 23.54 & 3.23 & 1.69 & 0.99 & 0.92 \\
\hline
3360 &  Timeout & 12.09 & 23.23 & Ε & Ε \\
\hline
6720 & Timeout & 48.72 & 51.90 & Ε & Ε \\
\hline
13440 &  Timeout & Timeout & Timeout & 49.06 & 39.51 \\
\hline
\end{tabular}
\caption{Χρόνοι για υβριδικά MPI--OpenMP}
\label{tab:timesMPIOpenMP}
\end{table}