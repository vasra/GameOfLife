\section{Επεξήγηση κώδικα}
Οι αλλαγές στον κώδικα είναι ελάχιστες. Συγκεκριμένα, έχουμε τις εξής επιπλέον μεταβλητές:

\begin{tcolorbox}
\begin{minted}{C}
int local_sum = 0, global_sum = 0;
\end{minted}
\end{tcolorbox}

\begin{multicols}{2}
Οι μεταβλητές αυτές αφορούν το άθροισμα του προσωπικού πλέγματος της κάθε διεργασίας, και το ολικό άθροισμα που προκύπτει από το Allreduce αντίστοιχα. Ένας pointer της \mintinline{C}{local_sum} περνιέται ως όρισμα στις συναρτήσεις \mintinline{C}{Next_generation_inner} και \mintinline{C}{Next_generation_outer} ώστε να υπολογίζεται επιτόπου το άθροισμα σε κάθε πέρασμα. Επίσης, στο κεντρικό loop έχουμε τον παρακάτω κώδικα, ο οποίος υλοποιεί το Allreduce κάθε δέκα επαναλήψεις.
\end{multicols}

\begin{tcolorbox}
\begin{minted}{C}
if (generation % 10 == 0)
    MPI_Allreduce(&local_sum, &global_sum, 1, MPI_INT, MPI_SUM, cartesian2D);
    
local_sum = 0;
\end{minted}
\end{tcolorbox}

\section{Μετρήσεις στην Αργώ}
\begin{multicols}{2}
Ακολουθούν οι πίνακες των χρόνων, επιτάχυνσης και αποδοτικότητας από τις μετρήσεις στην Αργώ. Στον πίνακα των χρόνων, με την τιμή «Ε» συμπληρώνονται τα κελιά για τα οποία δεν έχουν νόημα οι μετρήσεις, εφόσον εντοπίστηκε επιβράδυνση. Ο κύριος στόχος εδώ είναι να δούμε πώς επηρεάζει τις επιδόσεις το Allreduce κάθε δέκα επαναλήψεις -- και ο υπολογισμός του τοπικού αθροίσματος κάθε επανάληψη -- δεδομένου πως όλες οι υπόλοιπες συνθήκες είναι ίδιες με τα καθαρά MPI.
\end{multicols}

\begin{table}[H]
\centering
\small
\begin{tabular}{| l | c | c | c | c | c | c | c | c | c |}
\hline
\diagbox{Μέγεθος}{Διεργασίες} & 1 & 4 & 9 & 16 & 25 & 36 & 49 & 64 & 80\\
\hline
840 & 11.62 & 2.95 & 3.96 & Ε & Ε & Ε & Ε & Ε & Ε \\
\hline
1680 & 46.41 & 11.65 & 7.87 & 6.61 & 6.56 & 6.28 & Ε & Ε & Ε \\
\hline
3360 & 186.23 & 46.49 & \\
\hline
6720 & 743.16 &  \\
\hline
13440 & 2969.87 & \\
\hline
26880 &  \\
\hline
\end{tabular}
\caption{Χρόνοι για καθαρά MPI}
\label{tab:timesMPIrows}
\end{table}

\begin{table}[H]
\centering
\begin{tabular}{| l | c | c | c | c | c | c | c | c | c |}
\hline
\diagbox{Μέγεθος}{Διεργασίες} & 1 & 4 & 9 & 16 & 25 & 36 & 49 & 64 & 80\\
\hline
840 & 1 & \\
\hline
1680 & 1 & \\
\hline
3360 & 1 & \\
\hline
6720 & 1 & \\
\hline
13440 & \\
\hline
26880 & \\
\hline
\end{tabular}
\caption{Επιτάχυνση ($S = Ts / Tp$) για καθαρά MPI}
\label{tab:speedupMPIrows}
\end{table}

\begin{table}[H]
\centering
\begin{tabular}{| l | c | c | c | c | c | c | c | c | c |}
\hline
\diagbox{Μέγεθος}{Διεργασίες} & 1 & 4 & 9 & 16 & 25 & 36 & 49 & 64 & 80\\
\hline
840 & 1 & \\
\hline
1680 & 1 & \\
\hline
3360 & 1 & 1 & \\
\hline
6720 & 1 & \\
\hline
13440 & 1 & \\
\hline
26880 & \\
\hline
\end{tabular}
\caption{Αποδοτικότητα ($E = S / p$) για καθαρά MPI}
\label{tab:efficiencyMPIrows}
\end{table}
